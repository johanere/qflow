\documentclass[12pt]{article}

\usepackage{listings}


\title{qflow documenation}
\author{
       JN
}
\date{\today}


\begin{document}
\maketitle

\section{Terminology}
VMC simulation - A complete VMC simulation using

\section{qflow.training}
\textbf{ qflow.training.train( \textit{psi,
    H,
    sampler,
    iters,
    samples,
    gamma,
    optimizer,
    verbose,
    call\_backs,
    call\_back\_resolution
    })
}

\begin{enumerate}
\item    psi - Wavefunction class from qflow.wavefunctions 
\item    H - Hamiltonian class from qflow.hamiltonians 
\item    sampler - Sampler class from qflow.samplers
\item    iters - Number of complete VMC simulations used during training
\item    samples -
\item    gamma -
\item    optimizer - 
\item    verbose -
\item    call\_backs -
\item    call\_back\_resolution -
\end{enumerate}



\section{qflow.Hamiltonian}
\textbf{qflow.Hamiltonian.optimize\_wavefunction( \textit{ psi, 
	sampler, 
	iterations, 
	samples,
	optimizer, 
	gamma, 
	verbose 
	})
}

\begin{enumerate}
\item    psi - Wavefunction class from qflow.wavefunctions 
\item    sampler -  Sampler class from qflow.samplers
\item    iterations - Number of complete VMC simulations used during training
\item    samples - 
\item    optimizer -
\item    gamma -
\item    verbose - 
\end{enumerate}

optimize\_wavefunction uses a $0.2$ burn-in ratio to the total samples in each iteration. 
\end{document}
